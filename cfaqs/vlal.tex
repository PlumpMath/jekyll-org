% $Id$
\chapter{Variable-Length Argument Lists}	\label{chap:vlal}
% chapter 15: vlal

\begin{rawhtml}
<pre>$Id$</pre>
\end{rawhtml}

널리 알려지지는 않았지만, C 언어는 함수가 가변 인자를 (즉 인자의 갯수가 
정해지지 않은) 받을 수 있는 기능을 제공합니다.
Variable-length argument list(가변 인자 리스트)는, 드물기는 하지만,
\TT{printf}와 같은 함수들에게는 꼭 필요한 것입니다.
(variable-length argument list는 ANSI C 표준에서 공식적으로 지원하지만,
ANSI C 표준 이전에는 엄격히 말해서 정의되어 있지 않습니다.)

Variable-length argument list를 처리하는 방법은 상당히 기묘하기까지 합니다.
정식으로 varaible-length argument list는 fixed part(고정된 부분)와 
variable-length part(가변 길이)의
두 부분으로 나누어져 있습니다.  우리는 ``variable-length argument list의
variable-length part''라는 과장된 용어를 쓰고 있다는 사실을 발견했지만,
어쩔 수 없습니다 (혹시 여러분이 ``variadic''이나 ``varargs''라는 용어가
쓰이는 것을 보신 적이 있을지도 모르겠습니다: 두 가지 용어 모두
``having a variable number of arguments.\footnote{정해지지 않은 갯수의 인자를
가지는}''을 뜻합니다.  따라서 ``vararg function'' 또는 ``varargs argument''라고
말할 수도 있습니다.)

Variable-length argument list를 쓰는 방법은 크게 세 단계로 이루어 집니다.
먼저 \TT{va\_list} 타입의 특별한 pointer type을 써서 선언하고, 그 다음
\TT{va\_start}를 불러서 초기화합니다.
% TODO: errta 참고할 것.. variable argument list로 고쳐야 할 것 같은데..
% initialized to point to (the beginning of the argument list) by calling 
% va_start

\section{Calling Varargs Functions}

\begin{faq}
\Q{15.1}
	\TT{printf()}를 부르기 전에 \verb+#include <stdio.h>+를 쓰라고
	하더군요.  꼭 그럴 필요가 있을까요?
\A
	적절한 프로토타입(prototype)을 현재 영역(scope) 안에 포함시키기
	위해서 입니다.

	어떤 컴파일러에서는 가변 인자 리스트를 쓰는 함수에는 일반 방식과는
	다른 호출 순서(calling sequence)를 사용합니다.
% TODO: 다음 문장 번역.
	(It might do so if calls using variable-length argument lists were
	less efficient than those using fixed-length.)
	그러므로 함수의 프로토타입이 (즉 ``\verb+...+''를 포함한 선언)
	영역 안에 있어야 컴파일러가 가변 인자 리스트 처리 메커니즘을
	사용할 수 있습니다.

\R
	\cite{ansi} \S\ 3.3.2.2, \S\ 4.1.6 \\
	\cite{c89} \S\ 6.3.2.2, \S\ 7.1.7 \\
	\cite{rationale} \S\ 3.3.2.2, \S\ 4.1.6 \\
	\cite{hs} \S\ 9.2.4 \Page{268--9}, \S\ 9.6 \Page{275--6}
\end{faq}

\begin{faq}
\Q{15.2}
	\TT{printf()}에서 \verb+%f+가 \TT{float}과 \TT{double} 인자 모두에
	쓰일 수 있는 이유는 무엇인가요?
\A
	가변 인자 리스트에서 가변 인자 부분에는 ``default argument promotion''
	이 적용됩니다: 즉, \TT{char}와 \TT{short int} 타입은 \TT{int}로
	변경되며(promotion), \TT{float} 타입은 \TT{double} 타입으로
	변경됩니다.  (이 것은 함수의 프로토타입이 없거나 구 방식(old style)으로
	선언된 함수에서 일어나는 `promotion'과 같은 것입니다; 질문
	\ql{11.3}을 참고하기 바랍니다.) 그러므로
	\TT{printf}의 \verb+%f+ 포맷은 항상 \TT{double} 타입만 받아들이는 셈이
	됩니다.	(비슷한 이유에서 \verb+%c+, \verb+%hd+ 포맷은 항상 \TT{int}만을
	받아 들이게 됩니다.) \seealso{\ql{12.9}, \ql{12.13}}
\R
	\cite{ansi} \S\ 3.3.2.2 \\
	\cite{c89} \S\ 6.3.2.2 \\
	\cite{hs} \S\ 6.3.5 \page{177}, \S\ 9.4 \Page{272--3}
\end{faq}

\begin{faq}
\Q{15.3}
	\TT{n}이 \TT{long int}일 경우, 다음 코드에서 문제가 발생합니다:
\begin{verbatim}
  printf("%d", n);
\end{verbatim}
	\noindent 그렇지만 ANSI 함수 프로토타입을 제공했으니, 자동으로
	형 변환이 적용될 거라고 생각합니다.  무엇이 잘못되었나요?
\A
	함수가 가변인자를 받는다면, 프로토타입이 있더라도, 가변 인자에 대해서는
	어떠한 정보도 알지 못하며, 또한 알 수도 없습니다. 따라서, 일반적으로,
        가변 인자 리스트에서 가변 인자에 해당하는 부분은 그러한
        보호를 (type conversion) 받지 못합니다.
	컴파일러는 가변인자에 대해서는 implicit type conversion을
        수행하지 않으며, type이 서로 맞지 않는 경우 경고 해 줄 수 있습니다.
        따라서 프로그래머는 가변 인자로 들어간 인자가
	예상하는 타입과 같은지 확인해야 하며, 다를 경우에는 반드시
        캐스팅을 (explicit cast) 해 주어야 합니다:

	어떤 컴파일러들과 (gcc 포함), 어떤 \TT{lint} 프로그램은,
        format string이 문자열 상수(string literal)인 경우에 한해서,
        \TT{printf}와 같은 함수들에 전달된 가변 인자가 정말로 올바른 타입으로
        전달되었는지 검사해 주기도 합니다.

	\seealso{\ql{5.2}, \ql{11.3}, \ql{12.9}, \ql{15.2}}
\T
	\TT{gcc}의 경우, function attribute란 것으로 \TT{printf} 또는
        \TT{scanf}와 같은 함수의 가변 인자 부분을 검사해 줄 수 있습니다.
        자세한 것은 \TT{gcc} 매뉴얼을 참고 바랍니다.
\end{faq}

\section{Implementing Varargs Functions}

\begin{faq}
\Q{15.4}
	가변 인자를 받는 함수를 어떻게 만들 수 있을까요?
\A
	\verb+<stdarg.h>+ 헤더 파일에 있는 기능을 사용합니다.

	아래의 함수는 주어진 여러 개의 문자열을 붙여서 \TT{malloc()}으로 할당한
	메모리에 저장해서 리턴하는 함수입니다:

\begin{verbatim}
  #include <stdlib.h>     /* for malloc, NULL, size_t */
  #include <stdarg.h>     /* for va_ stuff */
  #include <string.h>     /* for strcat et al.  */

  char *vstrcat(char *first, ...)
  {
    size_t len;
    char *retbuf;
    va_list argp;
    char *p;

    if (first == NULL)
      return NULL;

    len = strlen(first);

    va_start(argp, first);

    while ((p = va_arg(argp, char *)) != NULL)
      len += strlen(p);

    va_end(argp);

    retbuf = malloc(len + 1);   /* +1 for trailing \0 */

    if (retbuf == NULL)
      return NULL;            /* error */

    (void)strcpy(retbuf, first);

    va_start(argp, first);      /* restart; 2nd scan */

    while ((p = va_arg(argp, char *)) != NULL)
      (void)strcat(retbuf, p);

    va_end(argp);

    return retbuf;
  }
\end{verbatim}
	\noindent (인자 리스트를 처음부터 다시 검색할 때에는 위에서처럼 
        \verb+va_start+를 다시 불러 주어야 합니다. 또한 \verb+va_end+가
        실제로 하는 일이 없을 수는 있지만, 이식성을 위해서 반드시 불러주어야 합니다.)


\noindent 사용법은 다음과 같습니다:
\begin{verbatim}
  char *str = vstrcat("Hello, ", "world!", (char *)NULL);
\end{verbatim}
	\noindent 마지막 인자를 캐스팅한 것을 꼭 주의깊게 보시기 바랍니다;
	질문 \ql{5.2}, \ql{15.3}을 참고하기 바랍니다.  (또한 이 함수를 부른 함수는
	이 함수가 리턴한 문자열을 \TT{free}시켜 주어야 할 책임이 있습니다.)

        위의 예는 모든 가변 인자의 타입이 \TT{char *}입니다. 이제, 가변 인자의 갯수와
        타입이 정해지지 않는 함수를 만들어 보겠습니다; 이 함수는 아주 간단한 기능만
        지원하는 \TT{printf}입니다. 특히 \verb+va_arg()+에서 원하는 타입을 미리
        지정한 부분에 대해 주의깊게 보시기 바랍니다.

        (아래 \TT{miniprintf} 함수는 질문 \ql{20.10}에서 쓴 \TT{baseconv}를
        씁니다. 또한 이 것은 \verb+INT_MIN+과 같은, 가장 작은 정수 값을 올바르게
        출력하기에 적당하지 않습니다.)

\begin{verbatim}
  #include <stdio.h>
  #include <stdarg.h>

  extern char *baseconv(unsigned int, int);

  void
  miniprintf(char *fmt, ...)
  {
    char *p;
    int i;
    unsigned u;
    char *s;
    va_list argp;

    va_start(argp, fmt);

    for (p = fmt; *p != '\0'; p++) {
      if (*p != '%') {
        putchar(*p);
        continue;
      }

      switch (*++p) {
      case 'c':
        i = va_arg(argp, int);
        /* not va_arg(argp, char); 질문 15.10 참고 */
        putchar(i);
        break;

      case 'd':
        i = va_arg(argp, int);
        if (i < 0) {
          /* XXX won't handle INT_MIN */
          i = -i;
          putchar('-');
        }
        fputs(baseconv(i, 10), stdout);
        break;
 
      case 'o':
        u = va_arg(argp, unsigned int);
        fputs(baseconv(u, 8), stdout);
        break;

      case 's': 
        s = va_arg(argp, char *);
        fputs(s, stdout);
        break;

      case 'u':
        s = va_arg(argp, unsigned int);
        fputs(baseconv(u, 10), stdout);
        break;

      case 'x':
        u = va_arg(argp, unsigned int);
        fputs(baseconv(u, 16), sttout);
        break;

      case '%':
        putchar('%');
        break;
      }      
    va_end(argp);
    }
  }
\end{verbatim}
	\noindent \seealso{\ql{15.7}}

\R
	\cite{kr2} \S\ 7.3 \page{155}, \S\ B7 \page{254} \\
        \cite{ansi} \S\ 4.8 \\
	\cite{c89} \S\ 7.8 \\
	\cite{rationale} \S\ 4.8 \\
	\cite{hs} \S\ 11.4 \Page{296--9} \\
	\cite{ctp} \S\ A.3 \Page{139--141} \\ 
	\cite{pcs} \S\ 11 \Page{184--5}, \S\ 13 \page{242}
\end{faq}

\begin{faq}
\Q{15.5}
	\TT{printf()}와 같이 포맷 문자열을 받아들여 처리하는 함수를
	만들어 그 처리를 \TT{printf()}에게 맡기고 싶습니다.
\A
	\TT{vprintf()}, \TT{vfprintf()}, \TT{vsprintf()} 함수를 쓰면
	됩니다. 이 함수들은, 각각 \TT{printf()}, \TT{fprintf()}, \TT{sprintf()}과
        같은 일을 하며, 단지 마지막 인자가 가변 인자인 대신 \verb+va_list+ 타입에
        대한 포인터를 받습니다.
        
        예를 들어, 아래의 \TT{error()} 함수는 에러 메시지를 받아들여
	그 메시지 앞에 ``error: ''를 덧붙이고 newline을 붙여서 출력해주는
	함수입니다:
\begin{verbatim}
  #include <stdio.h>
  #include <stdarg.h>

  void error(char *fmt, ...)
  {
    va_list argp;
    fprintf(stderr, "error: ");
    va_start(argp, fmt);
    vfprintf(stderr, fmt, argp);
    va_end(argp);
    fprintf(stderr, "\n");
  }
\end{verbatim}
	\noindent \seealso{\ql{15.7}}

\R
	\cite{kr2} \S\ 8.3 \page{174}, \S\ B1.2 \page{245} \\
	\cite{c89} \S\ 7.9.6.7, 7.9.6.8, 7.9.6.9 \\
	\cite{hs} \S\ 15.12 \Page{379--80} \\
	\cite{pcs} \S\ 11 \Page{186--7}
\end{faq}

\begin{faq}
\Q{15.6}
	\TT{scanf()}와 같은 기능을 하는 함수를 만들고 싶습니다.
\A
	\cite{c9x}는 \TT{vscanf()}와 \TT{vfscanf()}, \TT{vsscanf()}를
	지원할 것입니다.  (물론 당장 하기 위해서는 여러분 스스로가
	그러한 함수를 만들어야 합니다.)
\R
	\cite{c9x} \S\ 7.3.6.12--14
\T
	C99 표준은 \TT{vscanf()}, \TT{vfscanf()}, \TT{vsscanf()}를
        지원합니다. 따라서 질문 \ql{15.5}와 같은 방법으로 만들면 됩니다.
\R
	\cite{hs5} \S\ 15.12 \page{401}
        \cite{c99} \S\ 7.19.1, \S\ 7.19.6.9, \S\ 7.19.6.11, \S\ 7.19.6.14
% TODO: reference 추가.
\end{faq}

\begin{faq}
\Q{15.7}
	ANSI 이전의 컴파일러를 사용하고 있습니다.  \verb+<stdarg.h>+가 없는데
	어떻게 하죠?
\A
	\verb+<stdarg.h>+에 해당하는 오래된 헤더파일인 \verb+<varargs.h>+를
	쓰면 됩니다. 예를 들어, 질문 \ql{15.4}에서 만든 \TT{vstrcat}을,
        \verb+<varargs.h>+를 쓰도록 고치면 다음과 같습니다:
\begin{verbatim}
  #include <stdio.h>
  #include <varargs.h>
  #include <string.h>

  extern char *malloc();

  char *vstrcat(va_alist)
  va_dcl                       /* no semicolon */
  {
    int len = 0;
    char *retbuf;
    va_list argp;
    char *p;

    va_start(argp);

    while ((p = va_arg(argp, char *)) != NULL)
      len += strlen(p);

    va_end(argp);

    retbuf = malloc(len + 1);  /* +1 for trailing '\0' */
 
    if (retbuf == NULL)
      return NULL;             /* error */

    retbuf[0] = '\0';

    va_start(argp);            /* restart for second scan */

    while ((p = va_arg(argp, char *)) != NULL)
      strcat(retbuf, p);

    va_end(argp);

    return retbuf;
  }
\end{verbatim}
	\noindent (\verb+va_dcl+뒤에 세미콜론(`;')이 없는 것에 주의하기 바랍니다.
	그리고, 이 경우, 첫번째 인자에 대한 특별한 처리가 필요없습니다.) 또,
        \verb+<string.h>+을 포함시키는 대신, 직접 문자열 처리 관련 함수들을
        선언해 주어야 할 지도 모릅니다.

        만약, 시스템이 \TT{vfprintf}를 제공하며, \verb+<stdarg.h>+를 제공하지
        않을 경우, 아래에 (질문 \ql{15.5}에서 쓴) \verb+<varargs.h>+를 쓴
        \TT{error} 함수를 보입니다:
\begin{verbatim}
  #include <stdio.h>
  #include <varargs.h>

  void error(va_alist)
  va_dcl                  /* no semicolon */
  {
    char *fmt;
    va_list argp;
    fprintf(stderr, "error: ");
    va_start(argp);
    fmt = va_arg(argp, char *);
    vfprintf(stderr, fmt, argp);
    va_end(argp);
    fprintf(stderr, "\n");
  }
\end{verbatim}
	\noindent (\verb+<stdarg.h>+에서와는 달리, \verb+<varargs.h>+에서는
        모든 인자가 가변인자입니다. 따라서 \TT{fmt} 인자도, \verb+va_arg+를 써서
        얻어야 합니다.)
\R
	\cite{hs} \S\ 11.4 \Page{296--9} \\
	\cite{ctp} \S\ A.2 \Page{134--139} \\
	\cite{pcs} \S\ 11 \Page{184--5}, \S\ 13 \page{250}
\end{faq}

\section{Extracting Variable-Length Arguments}
\begin{faq}
\Q{15.8}
	함수에 몇 개의 인자가 전달되었는지를 정확히 알 수 있는 방법이
	있나요?
\A
	호환성있는 방법은 존재하지 않습니다.  어떤 오래된 시스템에서는
	비표준 함수인 \TT{nargs()}를 제공하기도 합니다.  그러나
	이 함수는 인자의 갯수를 리턴하는 게 아니라 전달된 word 갯수를
	리턴합니다. (구조체나 \TT{long int}, 실수는 여러 개의
	word로 이루어져 있는 경우가 대부분입니다.)

	가변 인자를 받아 처리하는 함수는 그 자체만으로 인자의 갯수를
	파악할 수 있어야 합니다.  \TT{printf} 계열의 함수들은 포맷 문자열에서
	(\TT{\%d}와 같은) 포맷 specifier를 보고 그 갯수를 파악합니다.
	(그렇기 때문에 \TT{printf()}에 전달된 인자의 갯수가 포맷 문자열과
	맞지 않을 경우에 오류를 일으킵니다.)
	또 다른 방법으로, 가변 인자가 모두 같은 타입일 경우, 마지막 인자를
	(0, -1, 또는 적절한 널 포인터와 같은) 어떤 특정한 값으로
	설정해서 인자의 갯수를 파악합니다 (질문 \ql{5.2}, \ql{15.4}에서
	\TT{execl()}과 \TT{vstrcat()} 함수의 사용법을 참고하시기 바랍니다).
	마지막으로, 인자의 갯수를 미리 파악할 수 있다면, 전체 인자의 갯수를
	인자로 전달하는 것도 좋은 방법입니다.  (although it's usually a
	nuisance for the caller to supply).
% TODO: 위 마지막 문장 번역.
\R
	\cite{pcs} \S\ 11 \Page{167--8}
\end{faq}

\begin{faq}
\Q{15.9}
	제 컴파일러는 다음과 같은 함수를 정의하면 에러를 냅니다.
\begin{verbatim}
  int f(...)
  {
  }
\end{verbatim}
\A
	표준 C에서는 \verb+va_start()+를 쓰려면 적어도 하나의 고정된 인자가
	있어야 한다고 말하고 있습니다. (어쨌든, 가변 인자의 갯수 또는 타입을 알려면,
        하나 이상의 인자가 필요합니다.) \seealso{\ql{15.10}}
\R
	\cite{ansi} \S\ 3.5.4, \S\ 3.5.4.3, \S\ 4.8.1.1 \\
	\cite{c89} \S\ 6.5.4, \S\ 6.5.4.3, \S\ 7.8.1.1 \\
	\cite{hs} \S\ 9.2 \page{263}
\end{faq}

\begin{faq}
\Q{15.10}
	가변 인자를 처리하는 함수에서 \TT{float} 인자를 처리하지 못합니다.
	왜 \verb+va_arg(argp, float)+을 쓰면 동작하지 않을까요?

\A
	가변 인자 리스트에서 가변 인자 부분은 항상 ``default argument
	promotion''이 적용됩니다: 즉 \TT{float} 타입의 인자들은 항상
	\TT{double}로 변환되며, \TT{char}나 \TT{short int}의 경우
	항상 \TT{int}로 변환됩니다.  

        \noindent 따라서, \verb+va_arg(argp, float)+은
	잘못된 코드이며, 대신 \verb+va_arg(argp,+\ \TT{double)}을
        써야 합니다.
	비슷한 이유로 \TT{char}, \TT{short}, \TT{int}를 받기 위해서는
	\verb+va_arg(argp, int)+를 써야 합니다.
	% TODO: 아래 문장 번역 -- 필요 없을 것 같다는 생각..
	% (For analogous reasons, the last "fixed" argument, as handed to
	% \verb+va_start()+, should not be widenable, either.)  
	\seealso{\ql{11.3}, \ql{15.2}}
\R
	\cite{c89} \S\ 6.3.2.2 \\
	\cite{rationale} \S\ 4.8.1.2 \\
	\cite{hs} \S\ 11.4 \page{297}
\end{faq}

\begin{faq}
\Q{15.11}
	함수 포인터를 \verb+va_arg()+로 얻을려고 하는데, 잘 안됩니다.
\A
	\TT{typedef}를 써서 함수 포인터를 새 타입으로 만들고 해 보기 바랍니다.

	\verb+va_arg()+와 같은 매크로는 함수 포인터와 같은 복잡한 타입을
	사용할 때, 곤란을 겪기도 합니다 (be stymied). 이 문제를 이해하기 위해,
        아래에 간단한 \verb+va_arg()+ 구현 예를 보입니다:
\begin{verbatim}
  #define va_arg(argp, type) \
    (*(type *)(((argp) += sizeof(type)) - sizeof(type)))
\end{verbatim}
	\noindent 위에서, \TT{argp}는 (즉, \verb+va_list+ 타입) 
        \TT{char *}입니다. 이 때, 만약 다음과 같이 불렀다면:
\begin{verbatim}
  va_arg(argp, int (*)())
\end{verbatim}
	\noindent \verb+va_arg+는 다음과 같이 확장됩니다:
\begin{verbatim}
  (*(int (*)() *)(((argp) += sizeof(int (*)())) -
      sizeof(int (*)())))
\end{verbatim}
	\noindent 위 결과를 자세히 보면 알겠지만, syntax error입니다. (첫번째
        \TT{(int (*)() *)}로 캐스트하는 것은 의미가 없습니다.)\footnote{이 확장이
          올바르려면 다음과 같아야 합니다:\\
        \TT{(*(int (**)())(((argp) += sizeof(int (*)())) - sizeof(int (*)())))}}
          
	만약, 함수 포인터를 다른 이름으로 \TT{typedef}했다면 모든 문제가 해결됩니다.
        주어진 함수 포인터를 다음과 같이 \TT{typedef}를 만들었다면:
\begin{verbatim}
  typedef int (*funcptr)();
\end{verbatim}
	\noindent 그리고 아래와 같이 불렀다면:
\begin{verbatim}
  va_arg(argp, funcptr)
\end{verbatim}
	\noindent 다음과 같이 확장됩니다:
\begin{verbatim}
  (*(funcptr *)(((argp) += sizeof(funcptr)) -
      sizeof(funcptr)))
\end{verbatim}
	\noindent 이 경우, 정상적으로 동작합니다.

	\seealso{\ql{1.13}, \ql{1.17} \ql{1.21}}
\R
	\cite{ansi} \S\ 4.8.1.2 \\
	\cite{c89} \S\ 7.8.1.2 \\
	\cite{rationale} \S\ 4.8.1.2
\end{faq}

\section{Harder Problems}
You can pick apart variable-length argument lists at run time, as we've seen.
But you can \EM{create} them only at compile time.  (We might say that strictly
speaking, there are no truly variable-length argument lists; every actual
argument list has some fixed number of arguments.  A vararg function merely
has the capability of accepting a different length of argument list with
each call.)  
If you want to call a function with a list of arguments created on the fly
at run time, you can't do so portably.

\begin{faq}
\Q{15.12}
	가변 인자를 받아서 다시 가변 인자를 처리하는 함수에 넘겨 줄 수 있을까요?
\A
	일반적으로, 불가능합니다.  이상적으로, 이 경우, \verb+va_list+를
        받는 함수를 만들어야 합니다.

        예를 들어, 치명적인(fatal) 에러 메시지를 출력하고 프로그램을 끝내는 함수인
        \TT{faterror}를 만든다고 가정해 봅시다. 여러분은 질문 \ql{15.5}에서 쓴
        \TT{error} 함수를 쓰길 원한다고 합시다:
\begin{verbatim}
  void faterror(char *fmt, ...)
  {
    error(fmt, ????);
    exit(EXIT_FAILURE);
  }
\end{verbatim}
	\noindent 위에서 보면 알겠지만, \TT{faterror}가 받은 가변 인자를
        \TT{error}에 전달할 방법이 없습니다.

        이 문제를 처리하기 위해서, 첫째, \TT{error} 함수를 분해해서 가변 인자 대신
        하나의 \verb+va_list+를 받는 \TT{verror} 함수를 만듭니다.
        (별로 어렵지 않습니다. 왜냐하면 \TT{verror}의 대부분은 \TT{error}와 같으며,
        일단 \TT{verror}를 만들고 나면, \TT{error} 함수는 \TT{verror}를 써서
        아주 쉽게 만들 수 있습니다.)
\begin{verbatim}
  #include <stdio.h>
  #include <stdarg.h>

  void verror(char *fmt, va_list argp)
  {
    fprintf(stderr, "error: ");
    vfprintf(stderr, fmt, argp);
    fpritnf(stderr, "\n");
  }

  void error(char *fmt, ...)
  {
    va_list argp;
    va_start(argp, fmt);
    verror(fmt, argp);
    va_end(argp);
  }
\end{verbatim}
	\noindent 위와 같이 만들었으면, 이제 \TT{faterror}를 다음과 같이
        만들 수 있습니다:
\begin{verbatim}
  void faterror(char *fmt, ...)
  {
    va_list argp;
    va_start(argp, fmt);
    verror(fmt, argp);
    va_end(argp);
    exit(EXIT_FAILURE);
  }
\end{verbatim}
	자세히 보면 \TT{error}와 \TT{verror}의 관계는 \TT{print}와 \TT{vprintf}의
        관계와 같습니다. Chris Torek씨가 제안한 것에 따르면, 여러분이 가변 인자를
        처리하는 함수를 만들때마다, 두 가지 버전을 제공하는 것이 좋습니다: 하나는
        (\TT{verror}와 같이) \verb+va_list+를 받아서 처리하는 함수와,
        (\TT{error}와 같이) 앞 함수의 간단한 wrapper 역할을 하는 함수, 두 가지입니다.
        이 방식의 한가지 단점은, \TT{verror}와 같은 함수는 가변 인자를 단 한번씩만
        scan할 수 있다는 것입니다; \verb+va_start+를 다시 부를 방법은 없습니다.

        만약, (\TT{faterror}와 같은) 가변 인자를 받아서 이 것을 가변 인자를 받는 다른
        함수에 전달하려고 하는데, (\TT{error}와 같이) \verb+va_list+를 받는 
        저수준 함수를 다시 만들 수 없다면, 이 문제를 해결하기 위한 portable한 방법은
        존재하지 않습니다. (이 문제를 시스템의 어셈블리 언어를 써서 해결할 가능성은
        있습니다. \seealso{\ql{15.13}})

        다음과 같은 방법은 동작하지 않습니다:
\begin{verbatim}
  void faterror(char *fmt, ...)
  {
    va_list argp;
    va_start(argp, fmt);
    error(fmt, argp);      /* WRONG */
    va_end(argp);
    exit(EXIT_FAILURE);
  }
\end{verbatim}
	\noindent \verb+va_list+ 자체는 가변 인자 리스트가 아닙니다; 내부적으로
        이 것은 가변 인자 리스트를 가리키는 포인터입니다. 따라서 \verb+va_list+를
        받는 함수는 가변 인자 리스트를 받는 함수가 아니며, 그 역도 성립하지 않습니다.

% 아래 paragraph 번역
        Another kludge that is sometimes used and that sometimes works even
        though it is grossly nonportable is to use a lot of \TT{int} arguments,
        hoping that there are enough of them and that they can somehow pass
        through pointer, floating-point, and other arguments as well:
\begin{verbatim}
  void faterror(fmt, a1, a2, a3, a4, a5, a6)
  char *fmt;
  int a1, a2, a3, a4, a5, a6;
  {
    error(fmt, a1, a2, a3, a4, a5, a6); /* VERY WRONG */
    exit(EXIT_FAILURE);
  }
\end{verbatim}
	위 예는, 이런 식으로 하지 말라는 것을 보여주기 위해 만든 것입니다.
        이 글에서 봤다는 이유로 위 코드를 쓰려고 하지 말기 바랍니다.
\end{faq}

\begin{faq}
\Q{15.13}
	런타임에 인자 리스트를 만들어서 함수를 부를 수 있는 방법은 없을까요?
\A
	호환성있는 방법도 없으며, 그런 일을 할 수 있다고 보장할 수도 없습니다.

	실제 인자 리스트를 처리하는 대신 generic (\verb+void *+) 포인터의 배열을
	넘겨주는 방법을 생각할 수 있습니다. 그리고 불려진 함수는 (\TT{main()}이
        \TT{argv} 인자를 처리하듯이) 이 배열 요소를 하나씩 조사해서 원하는
        정보를 얻을 수 있습니다. (물론, 이 방식은 여러분이 불려진 모든 함수를 직접
        제어할 수 있을 때에만 의미가 있습니다.) 

	\seealso{\ql{19.36}}
\end{faq}

%
% Local Variables:
% coding: utf-8
% fill-column: 78
% End:
%
