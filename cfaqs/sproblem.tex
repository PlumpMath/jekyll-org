% $Id$
\chapter{Strange Problems}	\label{chap:sproblem}
% chapter 16: sproblem

\begin{rawhtml}
<pre>$Id$</pre>
\end{rawhtml}

It's not even worth asking the rhetorical question, Have you ever had a
baffling bug that you just couldn't track down?  Of course you have;
everyone has.  C has a number of splendid ``gotcha!s'' lurking in wait for
the unwary;  this chapter discusses a few of them.  (In fact, any language
powerful enough to be popular probably has its share of surprises like these.)

\begin{faq}
\Q{16.1} % 16.1 COMPLETE
	왜 아래 loop는 항상 한 번만 실행될까요?
\begin{verbatim}
  for (i = start; i < end; i++);
  {
    printf("%d\n", i);
  }
\end{verbatim}
\A
	\TT{for}를 쓴 문장의 마지막 뒤에 실수로 붙인 세미콜론 때문에,
        \TT{for}가 반복할 문장(statement)은 ``null statement''가 됩니다.
        중괄호로 둘러싼 블럭은, 얼핏 보기에는 \TT{for}가 반복할 코드로 보이지만,
        사실은 \TT{for}와 무관한, 그냥 다음 statement일 뿐입니다.

        \seealso{\ql{2.18}}
\T
	위 코드를 다음과 같이 띄어쓰면 쉽게 이해할 수 있습니다:
\begin{verbatim}
  for (i = start; i < end; i++)
    ;

  {
    printf("%d\n", i);
  }
\end{verbatim}
\end{faq}

\begin{faq}
\Q{16.1b}
	문법 오류(syntax error)가 엄청나게 많이 발생했는데 도저히 그
	이유를 모르겠습니다.

\A
	먼저 헤더 파일과 소스 파일 전체에서
	\verb+#if+ / \verb+#ifdef+ / \verb+#ifndef+ / \verb+#else+ / 
        \verb+#endif+의
	쌍이 맞지 않는지 검사해보기 바랍니다.
	(\seealso{2.18, 10.9, 11.29})
\end{faq}

\begin{faq}
\Q{16.1c}
	왜 제 프로시져를 호출하는 부분이 동작하지 않을까요? 아무리 봐도
	컴파일러가 프로시져 호출 부분을 건너 뛰어 버리는 것 같습니다.

\A
	혹시 다음과 같이 코드를 작성하지 않았나요?
\begin{verbatim}
  myprocedure;
\end{verbatim}
	\noindent
	C 언어는 함수(function)만을 지원합니다.  그리고 함수에 전달되는
	인자가 하나도 없더라도 괄호로 둘러싼 `argument list'를 써 주어야
	합니다.  따라서 다음과 같이 써야 합니다:

\begin{verbatim}
  myprocedure();
\end{verbatim}
\end{faq}

\begin{faq}
\Q{16.2}
	파일의 첫 부분에서 이상한 syntax error가 납니다. 그런데 제가 보기에는 아무런
        문제가 없습니다.
\A
	질문 \ql{10.9}를 보기 바랍니다.
\end{faq}

\begin{faq}
\Q{16.3} % 16.3 COMPLETE
	동작하기도 전에 프로그램이 깨져버립니다.  (debugger로 한 스텝씩
	따라하면 \TT{main()}에 들어가기도 전에 죽어 버립니다.)
\A
	아마도 하나 이상의 (1KB 이상의)  매우 큰 local 배열을 썼을
	것입니다.  대부분의 시스템은 고정된 크기의 스택을 가지고 있고,
	동적으로 스택을 할당하는 (e.g.  Unix) 시스템이더라도 스택이
	갑자기 단 한번에 엄청나게 커지려고 하면 혼란을 겪게 됩니다.  매우 큰 배열이
	필요하다면 정적(static)으로 할당하는게 좋습니다 (함수를 호출할 때마다
	재귀 호출과 같은 이유로,
        새로운 배열이 필요하다면 \TT{malloc()}을 써서 할당하는 게 좋습니다.
	\seealso{\ql{1.31}}).

        또, 프로그램을 잘못 link했을 확률도 있습니다 (각각
        다른 컴파일러 옵션을 써서 만든 object module을 합쳤거나, 잘못된 dynamic
        library를 link했을 경우). 또는 run-time dynamic library linking이
        어떤 이유로 실패했거나, \TT{main}을 잘못 선언했을 수도 있습니다.

	(\seealso{\ql{11.12b}, \ql{16.4}, \ql{16.5}, \ql{18.4}})
\T
	물론, 위 답변에서 쓴 1KB의 크기는 시스템에 따라 매우 달라질 수 있습니다.
        이 글이 써진 시점이 조금 오래되었다는 것을 알기 바랍니다. 그러나,
        어떤 시스템이든 상관없이, 그 시스템에서 상대적으로 큰 크기의 스택을 쓰려 했을
        경우, 같은 현상이 일어날 수 있다는 것을 기억하기 바랍니다.
\end{faq}

\begin{faq}
\Q{16.4} % 16.4 COMPLETE
	제 프로그램은 정상적으로 동작하는데, 끝나기 직전에 박살납니다.
	즉 \TT{main()}의 마지막 문장을 실행한 다음에 깨지는 거죠.
	왜 그럴까요?
\A
	적어도 다음 세 가지 사항을 검사해보기 바랍니다:
        \begin{enumerate}
          \item \TT{main} 바로 위에 선언한 선언에서 세미콜론(`;')이 빠져 있으면,
            \TT{main} 함수의 리턴 타입이 잘못되어 structure를 리턴하도록
            선언되었을 가능성이 있습니다. 이 경우 run-time startup 코드와
            충돌하여, 이런 현상이 일어날 수 있습니다. 질문 \ql{2.18}, \ql{10.9} 참고
          \item \TT{setbuf}, \TT{setvbuf}에 전달한 버퍼가 \TT{main}에서 선언한
            local (그리고 automatic) 변수라면, \TT{main}이 끝나고 stdio 라이브러리가
            적절한 cleanup 코드를 수행할 때, 이 버퍼가 존재하지 않으므로 이러한
            문제가 발생할 수 있습니다.
          \item \TT{atexit}로 등록한 함수가 에러가 나서 이런 문제가 생길 수도
            있습니다. 아마도 \TT{main} 또는 다른 함수에 속해 있는 (local), 
            그래서 더 이상 존재하지 않는 데이터에 접근하려 했을 수 있습니다.
        \end{enumerate}

        \noindent (두 번째와 세 번째 문제는 질문 \ql{7.5}와 밀접한 관계가 있습니다;
        \seealso{\ql{11.16}})

\R
	\cite{ctp} \S\ 5.3 \Page{72--3}
\T
	또다른 경우, \TT{main}에서 선언한 local이며 automatic인 변수를 (대개는
        배열) 다룰 때, 이 변수의 범위를 벗어난 곳에 접근했을 (특히 write) 때,
        이러한 현상이 나타납니다. 물론 범위를 벗어난 곳에 접근하려 하는 것은
        그 자체가 ``undefined behavior''를 낳습니다.
\end{faq}

\begin{faq}
\Q{16.5} % 16.5 COMPLETE
	이 프로그램은 어떤 컴퓨터에서는 완벽하게 돌아가지만, 다른
	컴퓨터에서 돌리면 이상한 결과를 만들어 냅니다.  더욱 이상한 것은,
	가끔 debugging 목적으로 코드의 중간 중간에 출력하는 문장을 넣으면
        때때로 동작하기도 합니다. 왜 그럴까요?

\A
	이런 증상을 만들어 내는 이유는 많습니다.  아래에 일반적으로
	문제를 일으키는 경우가 있으니, 참고하시기 바랍니다:

	\begin{itemize}
	\item (uninitialized local variables\footnote{특히, stack-based
          시스템에서는, 초기화되지 않는 변수의 실제 값은, 그 당시 stack에 어떠한 값이
          있었느냐에 따라 달라집니다. 그렇기 때문에, 코드의 중간 중간 디버깅을 위해
          출력 문장을 넣었을 때, 동작할 수도 있습니다. 즉, \TT{printf}와 같은,
          코드의 덩치가 큰 함수를 실행하면, 스택에 있는 값들이 매우 달라질 수 있기
          때문입니다. \note 여기에서 말한 스택에 있는 값이란, 스택에 올바른 연산으로
        쌓여 있는 값이 아니라, 빈 스택에 있는 쓰레기 값을 의미합니다.})
          초기화되지 않은 변수 (\seealso{\ql{7.1}})
	\item integer overflow, 특히 16-bit 컴퓨터에서, 특히
		\verb+a * b / c+와 같은 연산을 할 때, 계산 도중 
                overflow가 일어날 때. (\seealso{\ql{3.14}})
	\item (undefined evaluation order) 평가 순서가 정의되지 않은 경우
          (질문 \ql{3.1}에서 \ql{3.4}까지 참고)
	\item external 함수의 선언이 생략되었을 경우.  특히 \TT{int}가 아닌
		다른 타입을 리턴하거나 ``narrow'' 또는 가변 인자를 가진 함수의 경우
		(질문 \ql{1.25}, \ql{11.3}, \ql{14.2}, \ql{15.1} 참고)
	\item 널 포인터를 dereference한 경우 (Chapter~\ref{chap:nullptr} 참고)

	\item \TT{malloc}/\TT{free}를 잘 못 쓴 경우, 특히 \TT{malloc}으로
		할당한 메모리가 0으로 초기화되었다고 가정한 경우나,
		이미 \TT{free}한 메모리를 계속 쓴 경우, 두 번 \TT{free}한
		경우, \TT{malloc}이 할당한 부분을 벗어나서 작업한 경우
		(\seealso{\ql{7.19}, \ql{7.20}})

	\item 일반적인 포인터 문제들 (\seealso{\ql{16.7} \ql{16.8}})

	\item \TT{printf()} 포맷과 인자가 서로 맞지 않은 경우, 특히
		\verb+%d+를 써서 \TT{long int}를 출력하려 한 경우
		(질문 \ql{12.7}, \ql{12.9} 참고)

	\item \TT{unsigned int}가 표현할 수 있는 범위 밖의 크기의 메모리를
		할당하려 한 경우, 특히 메모리에 제한 사항이 많은 컴퓨터에서.
                예를 들어 \TT{malloc(256 * 256 * sizeof(double))}과 같이.
		(\seealso{\ql{7.16}, \ql{19.23}})

	\item array overflow 문제들.  특히 작은 임시 버퍼를 쓸 때나
		\TT{sprintf()}를 써서 문자열을 만들 때.
                (\seealso{\ql{7.1}, \ql{12.21}, \ql{19.28}})

	\item \TT{typedef}된 타입을 특정 타입으로 잘못 예상하고
		코딩한 경우, 특히 \verb+size_t+에 대해. (질문 \ql{7.15} 참고)

	\item 실수(floating point) 처리 문제 (\seealso{\ql{14.1}, \ql{14.4}})

	\item 특정 시스템에서만 동작하게 되어 있는 교묘한(clever) 코드를
		사용한 경우
	\end{itemize}

	\noindent 올바른 함수 prototype을 사용한다면 이런 종류의 많은 문제를
	미리 잡아낼 수 있습니다; \TT{lint}를 쓰는 것도 좋은 방법입니다.
	\seealso{\ql{16.3}, \ql{16.4}, \ql{18.4}}
\T
	위에서 \TT{printf}에 특히, \TT{long int}를 잘못 전달했을 때, 그러한
        문제가 발생할 수 있다고 쓴 것은, (대부분의 16-bit 시스템처럼) 
        \TT{int}와 \TT{long int}가 실제 크기가 다른 시스템을 예상하고 말한 것입니다.
        (대부분의 32-bit 시스템처럼) 만약 \TT{int}와 \TT{long int}의 크기가
        같은 경우라면 이러한 문제가 자주 발생하지는 않습니다만, 어차피 잘못된 코드이기
        때문에 고쳐야 할 부분입니다. C99는 \TT{long long int}를 지원하므로, 
        \TT{long int}와 \TT{long long int}를 잘못 썼을 때 이런 문제가 발생할 수 있는
        확률이 높다고 할 수 있습니다.
\end{faq}

\begin{faq}
\Q{16.6}
	왜 이 코드가 동작하지 않을까요?:
\begin{verbatim}
  char *p = "hello, world!";
  p[0] = 'H';
\end{verbatim}

\A
	문자열 상수(string constant)도 상수입니다. 즉, 컴파일러는 문자열 상수를
        쓰기가 금지된 곳에 배치시킬 수 있으므로, 이것을 변경하려는 시도는 위험합니다.
        쓸(write) 수 있는 문자열이 필요하다면 쓸 수 있는 메모리를 
        배열을 선언하거나 \TT{malloc}으로 할당해 주어야 합니다.
        다음과 같이 해 보기 바랍니다:
\begin{verbatim}
  char a[] = "hello, world!";
\end{verbatim}
	\noindent 같은 이유에서, 다음과 같이,
        전통적으로 쓰는, UNIX \TT{mktemp} 쓰임새는 잘못된 것입니다:
\begin{verbatim}
  char *tmpfile = mktemp("/tmp/tmpXXXXXX");
\end{verbatim}
	\noindent 올바른 방법은 다음과 같습니다:
\begin{verbatim}
  char tmpfile[] = "/tmp/tmpXXXXXX";
  mktemp(tmpfile);
\end{verbatim}
	
	\seealso{\ql{1.32}}
\R
	\cite{ansi} \S\ 3.1.4 \\
	\cite{c89} \S\ 6.1.4 \\
	\cite{hs} \S\ 2.7.4 \Page{31--2}
\end{faq}

\begin{faq}
\Q{16.7}
	External structure를 푸는 코드를 받았습니다만, 실행하면 항상
        ``unaligned access''라는 에러가 납니다. 이게 무슨 뜻인가요?
        실제 코드는 다음과 같습니다:
\begin{verbatim}
  struct mystruct {
    char c;
    long int i32;
    int i16;
  };

  char buf[7], *p;
  fread(buf, 7, 1, fp);
  p = buf;
  s.c = *p++;
  s.i32 = *(long int *)p;
  p += 4;
  s.i16 = *(int *)p;
\end{verbatim}
\A
% TODO: 다음 문장 번역
	The problem is that you're playing too fast and loose with your
        pointers. 어떤 시스템들은 데이터 값들이 정확히 align되어 있어야만 그 값을
        access할 수 있습니다. 예를 들어, 2-byte \TT{short int}는 2의 배수 형태를
        띄는 주소값에서만 읽을 수 있으며, 4-byte \TT{long int}는 4의 배수 형태를
        띄어야만 access가 가능합니다. (질문 \ql{2.12} 참고). 
        (아무 곳이나 가리킬 수 있는) \TT{char *} 타입의 포인터를 강제로 \TT{int *}나
        \TT{long int *} 포인터로 바꿔 쓰면, 프로세서에게 multitype 값을 
        제대로 align되지 않은 곳에 access하게 만들게 됩니다.

        주어진 역할을 좀 더 올바르게 수행할 수 있는 방법은 다음과 같습니다:
\begin{verbatim}
  unsigned char *p = buf;

  s.c = *p++;

  s.i32 = (long *)p++ << 24;
  s.i32 |= (long *)p++ << 16;
  s.i32 |= (unsigned)(*p++ << 8);
  s.i32 |= *p++;

  s.i16 = *p++ << 8;
  s.i16 |= *p++;
\end{verbatim}
	\noindent This code also gives you control over byte order.
        (This example, though, assumes that a \TT{char} is 8 bits and the
        \TT{long int} and \TT{int} being unpacked from the ``external
        structure'' are 32 and 16 bits, respectively.)
        (비슷한 코드를 보여주는) 질문 \ql{12.42}를 보기 바랍니다. 

        \seealso{\ql{4.5}}
\R
	\cite{ansi} \S\ 3.3.3.2, \S\ 3.3.4 \\
        \cite{c89} \S\ 6.3.3.2, \S\ 6.3.4 \\
        \cite{hs} \S\ 6.1.3 \Page{164--5}
\end{faq}

\begin{faq}
\Q{16.8}
	``Segmentation violation''이나 ``Bus error''가 뭘까요?
	``core dump''는 또 뭘까요?

\A
	일반적으로 이런 메시지의 (또 memory-access violation 또는 protection 
        fault와 비슷한 모든 메시지) 뜻은, 여러분의 프로그램이 access할 수 없는 
        메모리 공간을 (잘못된 포인터 사용으로) access하려 했다는 것을 의미합니다.
        이런 메시지를 발생시킬 수 있는 몇가지 예는:
        \begin{itemize}
          \item 널 포인터를 잘못 썼을 때 (질문 \ql{5.2}, \ql{5.20} 참고)
          \item 초기화하지 않았거나, 잘못 aligne된 주소, 또 할당을 올바르지 않게
            한 포인터를 썼을 때 (질문 \ql{7.1}, \ql{7.2}, \ql{16.7} 참고)
          \item 다른 곳으로 배치된(relocated) 곳을 가리켜야 하지만, 그렇지 않는
            포인터를  (stale alias) 썼을 때 (질문 \ql{7.29} 참고)
          \item \TT{malloc}이 유지하는 내부 arena가 망가졌을 때 
            (질문 \ql{7.19} 참고)
          \item (\TT{const}로 선언되었거나, 문자열 상수처럼 --- 질문 \ql{1.32} 참고)
            읽기 전용인 값을 변경하려 할 때.
          \item 잘못 전달된 함수 인자, 특히 포인터 관련; 두 가지 예를 들면,
            \TT{scanf}를 잘 못 쓰거나 (질문 \ql{12.12}), \TT{fprintf}를
            잘 못 썼을 때 (첫번째 인자가 \TT{FILE *}라는 것을 꼭 검사하기 바람)
        \end{itemize}
        \noindent UNIX에서, 이런 모든 행위는 ``core dump''가 납니다; 즉,
        \TT{core}\footnote{Yes, the name ``core'' derives ultimately from old
          ferrite core memories.}라는 파일이 현재 디렉토리에 만들어 지며, 
        박살난 프로세스를 디버깅할 수 있도록 그 프로세스의 메모리 이미지가 저장되어
        있습니다.
        
        ``bus error''와 ``segmentation violation''의 차이는 중요하지 않습니다;
        각각 다른 상황에서 각각 다른 버전의 UNIX는 이러한 시그널들을 만들어 냅니다.
        간단히 말해, segmentation violation은, 아예 존재하지도 않는 메모리에
        access하려 했다는 것이고, bus error는 메모리를 잘못된 방법으로 access하려
        (대개 잘못된 align을 가지는 포인터, 질문 \ql{16.7} 참고) 했다는 것을 뜻합니다.
        
        \seealso{\ql{16.3}, \ql{16.4}}
\end{faq}

%
% Local Variables:
% coding: utf-8
% fill-column: 78
% End:
%
